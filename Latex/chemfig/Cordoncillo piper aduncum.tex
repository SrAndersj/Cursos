\documentclass[10pt,a4paper]{article}
\usepackage[utf8]{inputenc}
\usepackage[T1]{fontenc}
\usepackage{amsmath}
\usepackage{amsfonts}
\usepackage{amssymb}
\usepackage{graphicx}
\usepackage{chemfig}
\pagestyle{empty}
\begin{document}
	
%\chemfig[atom sep = 50pt]{a-b}

%\chemfig[bond style ={line width = 3pt,red}]{a-b}

%\chemfig{[atom sep = 50pt]}
	
	
	
%	\chemfig{[:90]*4(-([:60]*6(---=(*6(----(-[0])))))---)}
	
\chemname{\chemfig{[:90]([:60]*6(---(-[2]CH_3)=(*6(---(=[5]CH_2)-(-[0,1.73])(-[:-80]-[0,1.4](-[:-75]CH_3)(-[:-30]CH_3)-[:80](<:[-1]H))(<[5]H)))))}}{$\beta$-caryophyllene } \hspace{50mm}
\chemname{\chemfig{*6(-(-(<[:100,1.8]O?)(-[5])(-[7]))---?(-)--)}}{1,8 Cineole}
	
%	<bond>[<angle spec>,<length factor>,<other parameters>]
	
%	\chemfig{
%		CH
%		(-O-[:30]C(=[:90,.7]O)-[:-30]R_2)
%		(-[:90]CH_2-O-[:30]C(=[:90,.7]O)-[:-30]R_1)
%		(-[:-90]CH_2-O-[:30]C(=[:90,.7]O)-[:-30]R_3)

%	}

%setatomsep{50pt}\chemfig{CH([:45]-[0]O-[1]C(=[::+45]O)-[7]R_2)([:90]-CH_2 -[0]O-[1]C(=[::+45]O)-[7]R_1)([:-90]-CH_2 -[0]O-[1]C(=[::+45]O)-[7]R_3)}

%pdf2svg Halogenation.pdf dos.svg
%dvisvgm --no-fonts Halogenation.dvi
		 
	
\end{document}