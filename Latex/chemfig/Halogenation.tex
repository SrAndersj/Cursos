% !TeX spellcheck = es_ES
% !TeX TS-program = 
\documentclass[b5paper]{article}
\usepackage[utf8]{inputenc}
\usepackage[T1]{fontenc}
\usepackage{amsmath}
\usepackage{amsfonts}
\usepackage{amssymb}
\usepackage{graphicx}
\usepackage{chemfig}
\pagestyle{empty}
\begin{document}


%svg uno halogenation  

%
\schemestart
\chemfig{-[-1]C(-[-3])=C(-[1])-[-1]}  
\arrow{0}[,0]\+ 
\chemfig{X-X}
\arrow
\chemfig{-[-2]C(-[-2]X)(-[4])-C(-[6]X)(-[0])-[2]}
\schemestop


%
%\bigskip
%\bigskip
%svg dos

%\schemestart
%\chemfig{H-[-1]C(-[-3]H)=C(-[1]H)-[-1]H}  
%\arrow{0}[,0]\+ 
%\chemfig{Cl_2}
%\arrow
%\chemfig{H-[-2]C(-[-2]Cl)(-[4]H)-C(-[6]Cl)(-[0]H)-[2]H}
%\schemestop
%
%
%
%svg 3

%\schemestart
%\chemfig{*6(--=---)}  
%\arrow{0}[,0]\+ 
%\chemfig{Br_2}
%\arrow
%\chemfig{*6(--(-Br)-(-Br)---)}
%\schemestop


%svg 4








%enantiomeros
%\schemestart
%\chemfig{*6(--=---)}  
%\arrow{0}[,0]\+ 
%\chemfig{Br_2}
%\arrow
%\chemfig{*6(--(<:Br)-(<Br)---)}
%\arrow{0}[,0]\+ 
%\arrow{0}[,0]
%\chemfig{*6(--(<Br)-(<:Br)---)}
%\schemestop
%













%reaccion

\schemestart
\chemfig{[:90]-[1]@{aC}C(-[-1])=_[@{db}]C(-[3])-[1]}  
\hspace{1cm}
\chemfig{[:90]@{aCl2}\lewis{2:4:6:,X}-@{aCl}\lewis{0:2:6:,X}}
\arrow{->}
\chemfig{C*3(-C-\chemabove[3pt]{\lewis{0:4:,X}}{\scriptstyle\hspace{-1mm}+}-)}
\hspace{0.5cm}
\chemfig{+}
\hspace{0.5cm}
\chemfig{\chemabove[3pt]{\lewis{0:2:4:6:,X}}{\scriptstyle\hspace{-3mm}-}}
\schemestop
\chemmove{\draw [,,,,red](db).. controls +(100:5mm) and +(145:5mm)..(aCl);}
\chemmove{\draw [,,,,red](aCl).. controls +(65:-5mm) and +(170:-5mm)..(aC);}
\chemmove{\draw [,,,,red][shorten <=20pt,shorten >=4pt] (aCl.87).. controls +(-90:1cm) and +(120:1cm).. (aCl2);}

















%
%\bigskip
%\bigskip
%
\schemestart
\chemfig{C*3(-@{aC}C-@{aXX}\chemabove[3pt]{\lewis{0:4:,X}}{\scriptstyle\hspace{-1mm}+}-)}
\hspace{0.5cm}
\chemfig{+}
\hspace{0.5cm}
\chemfig{@{aX}\chemabove[3pt]{\lewis{0:2:4:6:,X}}{\scriptstyle\hspace{-3mm}-}}
\arrow{->}
\chemfig{-[-2]C(-[-2]\lewis{0:4:6:,X})(-[4])-C(-[6])(-[0])-[2]\lewis{0:2:4:,X}}
\schemestop
\chemmove{\draw [,,,,red][shorten <=6pt,shorten >=4pt](aX).. controls +(100:5mm) and +(14:5mm)..(aC);}
\chemmove{\draw [,,,,red][shorten <=6pt,shorten >=4pt](aC).. controls +(-210:5mm) and +(14:5mm)..(aXX);}
%
%








%	<bond>[<angle spec>,<length factor>,<other parameters>]

%	\chemfig{
%		CH
%		(-O-[:30]C(=[:90,.7]O)-[:-30]R_2)
%		(-[:90]CH_2-O-[:30]C(=[:90,.7]O)-[:-30]R_1)
%		(-[:-90]CH_2-O-[:30]C(=[:90,.7]O)-[:-30]R_3)

%	}





%pdf2svg Halogenation.pdf dos.svg
%dvisvgm --no-fonts Halogenation.dvi

\end{document}