\documentclass[10pt,a4paper]{book}
\usepackage[utf8]{inputenc}
\usepackage[T1]{fontenc}
\usepackage{amsmath}
\usepackage{amsfonts}
\usepackage{amssymb}
\usepackage{graphicx}
\usepackage[left=2.00cm]{geometry}
\usepackage{xymtex}
\usepackage{chemfig}


\begin{document}
Oxidation results in an increase the number of \chemfig{c-z} bonds 
z= mas electronegativo que el carbono
\bigskip
Hydrogeno -1 , heteroatomos +1 , 
	
	
\schemestart
\chemname{\chemfig{H-[:40](-[:320]H)(<:[:40]H)<[:120]H}}{most reduced form of carbon}
\arrow{->[$[O]$]}
\chemfig{H-[:40](-[:320]OH)(<:[:40]H)<[:120]H}
\arrow{->[$[O]$]}
\chemfig{H-[:40](=[:90]O)-[:320]H}
\arrow{->[$[O]$]}
\chemfig{H-[:40](=[:90]O)-[:320]OH}
\arrow{->[$[O]$]}
\chemname{\chemfig{O=C=O}}{most oxidized form of carbon}
\schemestop

\bigskip

Combustion

$CH_4+O_2$$\rightarrow$$CO_2+2H_2O+heat Energy$

\bigskip
Agentes reductores
\bigskip

$ Na^+$\chemfig{H-\chemabove{B}{\quad\scriptstyle-}(-[2]H)(-[6]H)-H}
\hspace{2em}
$ Li^+$\chemfig{H-\chemabove{Al}{\quad\scriptstyle-}(-[2]H)(-[6]H)-H}

\bigskip
Agentes oxidantes 
\bigski
\end{document}