\documentclass[10pt,a4paper]{article}
\usepackage[utf8]{inputenc}
\usepackage[T1]{fontenc}
\usepackage{amsmath}
\usepackage{amsfonts}
\usepackage{amssymb}
\usepackage{graphicx}
\usepackage{chemfig}
\begin{document}
%\chemfig{A-B-[1]C}
%\chemfig{A-[:45]-[::20]-[::45]B}

\chemfig{H-[:30]O-[:-30]H}

%\chemfig{A-B(-[1]w-X)-C}

\chemfig{R-C(=[::+60]O)-[::-60]O-[::-60]C(=[::+60]O)-[::-60]R}


\chemfig{[:90]R-C(=[::+60]O)-[::-60]O-[::-60]C(=[::+60]O)-[::-60]R}


\chemfig{A*5(-B=C-D-E=)}


\chemfig{*5(-=--=)}



\chemfig{**6(------)}\quad
\chemfig{**[30,330]5(-----)}\quad
\chemfig{**[0,270,dash pattern=on 2pt off 2pt]4(----)}

rotation

\chemfig{A*6(------)}

\chemfig{[:60]A*6(------)}


ring 4
\linebreak
\chemfig{A-[:25]B*4(----)}

\chemfig{X*6(-=-(-A-B=C)=-=-)}

\chemfig{*5(--*6(-*4(-*5(----)--)----)---)}

\chemfig{A-B*5(-C-D*5(-X-Y-Z-)-E-F-)}





\schemestart
\chemfig{@{a1}=_[@{db}::30]-[::-60]\lewis{2,X}}
\arrow{<->}
\chemfig{\chemabove{\vphantom{X}}{\ominus}-[::30]=_[::-60]
	\chemabove{X}{\scriptstyle\oplus}}
\schemestop
\chemmove{\draw(db).. controls +(100:5mm) and +(145:5mm)..(a1);}





\schemestart[][west]
\chemfig{-[:45]C=C(-[:-45])-[:45]}
\arrow{0}[,0]\+
\chemfig{\lewis{246,Br}-\lewis{026,Br}}
\schemestop

\newpage

svg uno halogenation  


\schemestart
\chemfig{-[-1]C(-[-3])=C(-[1])-[-1]}  
\arrow{0}[,0]\+ 
\chemfig{X-X}
\arrow
\chemfig{-[-2]C(-[-2]X)(-[4])-C(-[6]X)(-[0])-[2]}
\schemestop

\bigskip
\bigskip
svg dos

\schemestart
\chemfig{H-[-1]C(-[-3]H)=C(-[1]H)-[-1]H}  
\arrow{0}[,0]\+ 
\chemfig{Cl_2}
\arrow
\chemfig{H-[-2]C(-[-2]Cl)(-[4]H)-C(-[6]Cl)(-[0]H)-[2]H}
\schemestop



svg 3

\schemestart
\chemfig{*6(--=---)}  
\arrow{0}[,0]\+ 
\chemfig{Br_2}
\arrow
\chemfig{*6(--(-Br)-(-Br)---)}
\schemestop


svg 4

\schemestart
\chemfig{*6(--=---)}  
\arrow{0}[,0]\+ 
\chemfig{Br_2}
\arrow
\chemfig{*6(--(<:Br)-(<Br)---)}
\arrow{0}[,0]\+ 
\arrow{0}[,0]
\chemfig{*6(--(<Br)-(<:Br)---)}
\schemestop



\part{flecha roja desde enlace sencillo hasta cloro}

\schemestart
\chemfig{[:90]@{aCl2}\lewis{2:4:6:,Cl}-@{ob}\lewis{0:2:6:,Cl}}
\schemestop
\chemmove{\draw [,,,,red][shorten <=20pt,shorten >=4pt] (ob.87).. controls +(-90:1cm) and +(120:1cm).. (aCl2);}





\schemestart
\chemfig{@{a1}=_[@{db}::30]-[::-60]\lewis{2,X}}
\arrow{<->}
\chemfig{\chemabove{\vphantom{X}}{\ominus}-[::30]=_[::-60]
	\chemabove{X}{\scriptstyle\oplus}}
\schemestop
\chemmove{\draw(db).. controls +(100:5mm) and +(145:5mm)..(a1);}



\part{flecha roja desde doble enlace a cloro molecular y luego desde enlace sencillo hasta cloro}

\schemestart
\chemfig{[:90]-[1]@{aC}C(-[-1])=_[@{db}]C(-[3])-[1]}  
\hspace{1cm}
\chemfig{[:90]@{aCl2}\lewis{2:4:6:,Cl}-@{aCl}\lewis{0:2:6:,Cl}}
\arrow{->}
\chemfig{C*3(-C-\chemabove[3pt]{\lewis{0:4:,Cl}}{\scriptstyle\hspace{-1mm}+}-)}
\schemestop
\chemmove{\draw(db)[,,,,red].. controls +(100:5mm) and +(145:5mm)..(aCl);}
\chemmove{\draw(aCl)[,,,,red].. controls +(65:-5mm) and +(170:-5mm)..(aC);}
\chemmove{\draw [,,,,red][shorten <=20pt,shorten >=4pt] (aCl.87).. controls +(-90:1cm) and +(120:1cm).. (aCl2);}


control t
control u

\title{ pdf2svg Halogenation.pdf dos.svg
}

\end{document}